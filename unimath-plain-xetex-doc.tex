%% The document of unimath-plain-xetex
%% ******************************************************
%% * This work may be distributed and/or modified under *
%% * the conditions of the LaTeX Project Public License *
%% *     http://www.latex-project.org/lppl.txt          *
%% * either version 1.3c of this license or any later   *
%% * version.                                           *
%% ******************************************************
\def\mainfontname{erewhon}
\def\sansfontname{Fira Sans Book}
\def\monofontname{Source Code Pro}
\def\mathfontname{Erewhon Math}
\input unimath-plain-xetex
\unicodeprimesoff
\betweenprimeskip=-1mu\relax

\def\xetex{X\kern-.125em\lower.5ex\hbox{\char"018E}\kern-.1667em%
  T\kern-.1667em\lower.5ex\hbox{E}\kern-.125emX}
\let\XeTeX\xetex
\let\tex\TeX
\def\latex{L\kern-.24em{\setbox0=\hbox{T}\vbox to \ht0{\hbox{\sevenrm A}\vss}}%
  \kern-.12em\TeX}
\def\umpx{unimath-plain-\xetex}
\def\biggskip{\vskip24pt plus 8pt minus 4pt}
\def\pkg#1{{\tensf #1}}
%\catcode`\@=11 \@sizeat{40pt}\the\sf@size \catcode`\@=12
\font\hugesans="\sansfontname" at 35pt
\font\sevenlmmath="Latin Modern Math:script=math,+ssty=0" at 7pt
\makefontcmdcompatible{ten}

\def\ttverbatim{\begingroup
  \catcode`\\=12 \catcode`\{=12 \catcode`\}=12 \catcode`\$=12
  \catcode`\&=12 \catcode`\#=12 \catcode`\%=12 \catcode`\~=12
  \catcode`\_=12 \catcode`\^=12 \obeyspaces \obeylines \tt}
\outer\def\begintt{$$\let\par=\endgraf \ttverbatim \parskip=0pt
  \catcode`\|=0 \rightskip-2pc \ttfinish}
{\catcode`\|=0 |catcode`|\=12 % | is temporary escape character
  |obeylines % end of line is active
  |gdef|ttfinish#1^^M#2\endtt{#1|vbox{#2}|endgroup$$}}
\def\charhex{\char"}
\catcode`\"=\active
{\obeylines \gdef"{\ttverbatim \spaceskip.5em \let^^M=\  \let"=\endgroup}}
\def\<#1>{$\langle${\it#1\/}$\rangle$}

\long\def\scalehbox#1#2#3{%
  \leavevmode
  \setbox0\hbox{{#3}}%
  \setbox1\hbox{%
    \special{pdf:btrans}%
    \special{x:scale #1 #2}%
    \hbox to 0pt{\copy0\hss}%
    \special{pdf:etrans}%
  }%
  \ht1#2\ht0
  \dp1#2\dp0
  \hbox to#1\wd0{\box1\kern#1\wd0\hss}%
}

\newdimen\paperwd
\newdimen\paperht
\newdimen\hmarginwd
\newdimen\vmarginht
\hsize=12cm
\vsize=16cm
\hmarginwd=3cm
\vmarginht=3cm
\hoffset=\dimexpr\hmarginwd-1in\relax
\voffset=\dimexpr\vmarginht-1in\relax
\paperwd=\dimexpr2\hmarginwd+\hsize\relax
\paperht=\dimexpr2\vmarginht+\vsize\relax
\special{papersize=\the\paperwd,\the\paperht}
\newcount\seccnt
\seccnt=0
\long\def\section#1{\penalty-1000\bigskip%
  \advance\seccnt1\relax\noindent{\twelvebf%
    \hbox to 0pt{\hss\twentybf\the\seccnt\kern12pt}#1}%
  \par\medskip\penalty1000}
\parindent2pc

% title
\centerline{\twentyrm The 
  \scalehbox{0.5}{0.5}{%
  \special{pdf:code q 1 Tr}%
  {\hugesans\umpx}%
  \special{pdf:code Q}}
  package}
\biggskip

\centerline{Zhang Tingxuan}
\medskip

\centerline{2022/11/23\quad Version 0.0γ}
\biggskip

\centerline{\bf Abstract}
\medskip
{\sf\umpx} package provides OpenType math font support in {\it plain \tex\/} 
format. The {\sf\umpx} package needs \xetex.

\section{How to use this package?}
Please notice again that you're using {\it plain\/} format but not 
\latex\ format. If you are using \latex\ format, please use 
\pkg{unicode-math} package instead.

In your document, write
\begintt
\input unimath-plain-xetex
\endtt
Then compile your document with "xetex", you can get OpenType math support 
in your document. The package will set the math font ``Latin Modern Math'' 
with ``Latin Modern'' text fonts in default. To change the font, you can 
define some names before loading the package. For example,
\begintt
\def\mainfontname{TeX Gyre Termes}
\def\sansfontname{TeX Gyre Heros}
\def\monofontname{TeX Gyre Cursors}
\def\mathfontname{TeX Gyre Termes Math}
\input unimath-plain-xetex
\endtt
Your text fonts will be set in the first 3 lines and your math font will 
be set in the fourth line.

Currently, the package supports only font family names to use, if 
you want to use the file names, you can revise the code in 
"unimath-plain-xetex.tex".



\section{Text font commands}
The package provides text font commands in the format of 
$$\hbox{"\"\<pt-size>\<family>\<series>\<shape>}$$
such as "\tensfbfit", "\twelvebf", etc. The packages provides 
font commands in size of 5\,pt, 7\,pt, 9\,pt, 10\,pt, 12\,pt and 20\,pt, 
such as "\fiverm", "\sevensf", "\twelvett", etc.

Take ten point as an example,
$$\vbox{\openup10pt%
  \halign{ \hfil#\hfil \cr
  {\twelverm Serif} \cr 
    \vbox{\openup-8pt\halign{#&\quad#&\quad#\cr 
                    & Upright  & {\tenit Italic} \cr
      Medium        & "\tenrm" & "\tenit"        \cr
      {\tenbf Bold} & "\tenbf" & "\tenbfit"      \cr}} \cr
  {\twelvesf Sans} \cr 
    \vbox{\openup-8pt\halign{#&\quad#&\quad#\cr 
                      & {\tensf Upright}  & {\tensfit Italic} \cr
      {\tensf Medium} & "\tensf"   & "\tensfit"        \cr
      {\tensfbf Bold} & "\tensfbf" & "\tensfbfit"      \cr}} \cr
  {\twelvett Typewritter} \cr
    \vbox{\openup-8pt\halign{#&\quad#&\quad#\cr 
                      & {\tentt Upright}  & {\tenttit Italic} \cr
      {\tentt Medium} & "\tentt"   & "\tenttit"        \cr
      {\tenttbf Bold} & "\tenttbf" & "\tenttbfit"      \cr}} \cr}}$$
The font commands can be used as those provided in "plain.tex", for example,
"{\tenbfit ABC}" yields {\tenbfit ABC}.

But if you would't like to remember that many commands, you can write
$$\hbox{\tentt"\makefontcmdcompatible"\{\<pt-size>\}}$$
after loading the package. For example, if you write
\begintt
\makefontcmdcompatible{ten}
\endtt
the \<family>-\<series>-\<shape> order of "\ten"\<some> commands 
can be write randomly: writing "\tenbfsfit" is the same as "\tensfbfit".

You can also get more text font commands through the "\genfontcmd" command:
$$\hbox{\tentt"\genfontcmd"\{\<pt-size>\}\{\<dimension>\}}$$
For example, 
\begintt
\genfontcmd{fortyfour}{44pt}
\genfontcmd{verytiny}{2bp}
\endtt
will make commands like "\fortyfourrm" and "\verytinysfbfit" available.



\section{Math font commands}
%Traditional plain \tex\ uses "\fam" mechanism to change the font of letters 
%and numbers in math mode. But using the "\fam" mechanism to change the face 
%of a math symbol in an OpenType math font is almost impossible. For example, 
%the roman ``${\rm A}$'' is encoded U+0041, but the italic math ``$A$'' is 
%encoded U+1D434.But this problem can be solved through redefining the math 
%character.
You can input math formulae just like using traditional plain \tex. But 
OpenType math font is loaded. For example, "$a{\bf0}={\bf0}$" yields 
$a{\bf0}={\bf0}$. Available math font commands are listed below:
\begintt
\rm, \bf, \it, \bfit,\sf, \sfbf, \sfit, \sfbfit,
\tt, \cal, \calbf, \bb, \bbit, \frak, \frakbf
\endtt
The \<family>-\<series>-\<shape> order of these commands allows being random; 
"cal" can be replaced by "scr".

This package uses "unicode-math-symbols.tex" to generate math symbol commands, 
the source file can be found in \pkg{unicode-math} package. To find all of 
the math symbol commands, you can execute
\begintt
texdoc unimath-symbols
\endtt
in Terminal.

You can also input Unicode math characters in your document's source file. 
For example, "$∫_a^b y\,dx$" yields $∫_a^b y\,dx$ and "$$∫_a^b y\,dx.$$" yields 
$$∫_a^b y\,dx.$$



\section{Primes ($'$)}
The package changed ``"'"'' command in math mode, which yields prime(s) 
in superscript. After loading this package, the primes produced by ``"'"'' 
will be turned into Unicode primes:
$$\vbox{\openup2pt\halign{#\hfil\quad&\quad#\hfil\quad&\quad#\hfil\cr
{\it Input} & {yields} & {\it Unicode slot}\cr
"$'$"     & $^{\hbox{\sevenlmmath\charhex2032}}$ & U+2032 \cr
"$''$"    & $^{\hbox{\sevenlmmath\charhex2033}}$ & U+2033 \cr
"$'''$"   & $^{\hbox{\sevenlmmath\charhex2034}}$ & U+2034 \cr
"$''''$"  & $^{\hbox{\sevenlmmath\charhex2057}}$ & U+2057 \cr
"$'''''$" & $^{\hbox{\sevenlmmath\charhex2032
  \hskip-.15em\charhex2032
  \hskip-.15em\charhex2032
  \hskip-.15em\charhex2032
  \hskip-.15em\charhex2032}}$
  & $(\hbox{U+2032})\times5$\cr
$\vdots$ & $\vdots$ & $\vdots$ \cr
$(\hbox{"'"})\times N$ & $(^{\hbox{\sevenlmmath\charhex2032}})\times N$ &
$(\hbox{U+2032})\times N$\hbox to 20pt{\quad$(N>4)$\hss} \cr
}}$$
However, some OpenType math fonts don't contain some of the characters above. 
For example, there is no U+2032, U+2033 or U+2034 in Erewhon Math. 
When using such fonts, you can ``turn off'' the Unicode primes easily 
by inputing the following line {\it after} loading the package:
\begintt
\unicodeprimesoff
\endtt
Then when you input ``"'"'' $n$ times you will get $n$ primes, 
and each prime's Unicode character slot is U+2032, even when $n\leq 4$.

If the primes contain more than one single prime encoded U+2032, 
a negative math skip will inserted between every two single primes. 
This math skip is defined as "\betweenprimeskip" and its default value 
is "-2.7mu", which is proper for Latin Modern Math. To change the value 
of "\betweenprimeskip" is just like change any math skip, for example,
\begintt
\betweenprimeskip=-1mu
\endtt
This line should also be written {\it after} loading the package.

\bye