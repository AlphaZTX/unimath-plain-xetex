% unimath-plain-xetex.tex
\catcode`\@=11

\newdimen\@tempdima
\newdimen\@tempdimb
\newdimen\@tempdimc
\newdimen\@tempdimd

% text font
%\gdef\mainfontname{Latin Modern Roman}
\ifdefined\mainfontname\else
  \gdef\mainfontname{Latin Modern Roman}
\fi
\ifdefined\sansfontname\else
  \gdef\sansfontname{Latin Modern Sans}
\fi
\ifdefined\monofontname\else
  \gdef\monofontname{Latin Modern Mono}
\fi
\ifdefined\mathfontname\else
  \gdef\mathfontname{Latin Modern Math}
\fi
\ifcsname XeTeXversion\endcsname
  \ifdefined\textfontopt\else
    \def\textfontopt{mapping=tex-text}
  \fi
\else
  \errmessage{umath-plain-xetex Error: Needs XeTeX!}
\fi

\font\tenrm     =  "\mainfontname/R:\textfontopt" at 10pt
\font\tenit     =  "\mainfontname/I:\textfontopt" at 10pt
\font\tenbf     =  "\mainfontname/B:\textfontopt" at 10pt
\font\tenbfit   = "\mainfontname/BI:\textfontopt" at 10pt
% scaling sf and tt
\newdimen\sf@size
\newdimen\tt@size

\font\tensf@test = "\sansfontname/R:\textfontopt" at 10pt
\font\tentt@test = "\monofontname/R:\textfontopt" at 10pt

\@tempdima\fontdimen5\tenrm
\@tempdimb\fontdimen5\tensf@test
\gdef\sf@innerratio{\numexpr\dimexpr256\@tempdima/\@tempdimb\relax}
\@tempdimc\fontdimen5\tentt@test
\gdef\tt@innerratio{\numexpr\dimexpr256\@tempdima/\@tempdimc\relax}
\let\tensf@test\relax \let\tentt@test\relax

\def\@sizeat#1{\@tempdimd#1\relax%
  \sf@size=\dimexpr\sf@innerratio\@tempdimd/256\relax
  \tt@size=\dimexpr\tt@innerratio\@tempdimd/256\relax
}
\@sizeat{10pt}
\font\tensf     =  "\sansfontname/R:\textfontopt" at \sf@size
\font\tensfit   =  "\sansfontname/I:\textfontopt" at \sf@size
\font\tensfbf   =  "\sansfontname/B:\textfontopt" at \sf@size
\font\tensfbfit = "\sansfontname/BI:\textfontopt" at \sf@size
\font\tentt     =  "\monofontname/R:\textfontopt" at \tt@size
\font\tenttit   =  "\monofontname/I:\textfontopt" at \tt@size
\font\tenttbf   =  "\monofontname/B:\textfontopt" at \tt@size
\font\tenttbfit = "\monofontname/BI:\textfontopt" at \tt@size
% other sizes
\def\genfontcmds#1#2{\@sizeat{#2}%
  \expandafter\font\csname     #1rm\endcsname =  "\mainfontname/R:\textfontopt" at #2
  \expandafter\font\csname     #1it\endcsname =  "\mainfontname/I:\textfontopt" at #2
  \expandafter\font\csname     #1bf\endcsname =  "\mainfontname/B:\textfontopt" at #2
  \expandafter\font\csname   #1bfit\endcsname = "\mainfontname/BI:\textfontopt" at #2
  \expandafter\font\csname     #1sf\endcsname =  "\sansfontname/R:\textfontopt" at \sf@size
  \expandafter\font\csname   #1sfit\endcsname =  "\sansfontname/I:\textfontopt" at \sf@size
  \expandafter\font\csname   #1sfbf\endcsname =  "\sansfontname/B:\textfontopt" at \sf@size
  \expandafter\font\csname #1sfbfit\endcsname = "\sansfontname/BI:\textfontopt" at \sf@size
  \expandafter\font\csname     #1tt\endcsname =  "\monofontname/R:\textfontopt" at \tt@size
  \expandafter\font\csname   #1ttit\endcsname =  "\monofontname/I:\textfontopt" at \tt@size
  \expandafter\font\csname   #1ttbf\endcsname =  "\monofontname/B:\textfontopt" at \tt@size
  \expandafter\font\csname #1ttbfit\endcsname = "\monofontname/BI:\textfontopt" at \tt@size
}
\genfontcmds{five}{5pt}
\genfontcmds{seven}{7pt}
\genfontcmds{nine}{9pt}
\genfontcmds{twelve}{11.95pt}
\genfontcmds{twenty}{20pt}
\def\makefontcmdcompatible#1{%
  \expandafter\let\csname #1itbf\expandafter\endcsname\csname #1bfit\endcsname
  \expandafter\let\csname #1itsf\expandafter\endcsname\csname #1sfit\endcsname
  \expandafter\let\csname #1bfsf\expandafter\endcsname\csname #1sfbf\endcsname
  \expandafter\let\csname #1sfitbf\expandafter\endcsname\csname #1sfbfit\endcsname
  \expandafter\let\csname #1bfsfit\expandafter\endcsname\csname #1sfbfit\endcsname
  \expandafter\let\csname #1bfitsf\expandafter\endcsname\csname #1sfbfit\endcsname
  \expandafter\let\csname #1itsfbf\expandafter\endcsname\csname #1sfbfit\endcsname
  \expandafter\let\csname #1itbfsf\expandafter\endcsname\csname #1sfbfit\endcsname
  \expandafter\let\csname #1ittt\expandafter\endcsname\csname #1ttit\endcsname
  \expandafter\let\csname #1bftt\expandafter\endcsname\csname #1ttbf\endcsname
  \expandafter\let\csname #1ttitbf\expandafter\endcsname\csname #1ttbfit\endcsname
  \expandafter\let\csname #1bfttit\expandafter\endcsname\csname #1ttbfit\endcsname
  \expandafter\let\csname #1bfittt\expandafter\endcsname\csname #1ttbfit\endcsname
  \expandafter\let\csname #1itttbf\expandafter\endcsname\csname #1ttbfit\endcsname
  \expandafter\let\csname #1itbftt\expandafter\endcsname\csname #1ttbfit\endcsname
}

% math font
%
% In plain format, \fam0 is "rm"; \fam1 is "normal"; \fam2 is "cal"; 
%                  \fam3 is "op"; \fam4 is "it"; \fam5 is "sl";
%                  \fam6 is "bf"; \fam7 is "tt".
% Families defined through \newfam: 
%               \itfam (4), \slfam (5), \bffam (6), \ttfam (7).
\newfam\unimathfam % \unimathfam = 8
\newfam\textfam    % \textfam = 9, for text in math mode ("math text")

\font   \tenmath = "\mathfontname:script=math" at 10pt
\font \sevenmath = "\mathfontname:script=math,+ssty=0" at 7pt
\font  \fivemath = "\mathfontname:script=math,+ssty=1" at 5pt
\textfont         \unimathfam =   \tenmath
\scriptfont       \unimathfam = \sevenmath
\scriptscriptfont \unimathfam =  \fivemath

% load unicode-math-table
\let\mathalpha\mathord
\def\mathfence{F}
\def\mathaccentwide{Awo}
\def\mathaccentoverlay{Awo}
\def\mathover{O}
\def\mathunder{U}
\def\mathbotaccent{bA}
\def\mathbotaccentwide{bAw}
\def\UnicodeMathSymbol#1#2#3#4{% #1: char slot; #2: cmd; #3: \mathord, etc.
  \ifx#3\mathord
    \Umathchardef #2 = 0 \unimathfam #1 
    \Umathcode    #1 = 0 \unimathfam #1 
  %\else\ifx#3\mathalpha
  %  \Umathchardef #2 = 0 \unimathfam #1 
  %  \Umathcode    #1 = 0 \unimathfam #1 % a \fi in the end
  \else\ifx#3\mathop
    \Umathchardef #2 = 1 \unimathfam #1 
    \Umathcode    #1 = 1 \unimathfam #1 
  \else\ifx#3\mathbin
    \Umathchardef #2 = 2 \unimathfam #1 
    \Umathcode    #1 = 2 \unimathfam #1 
  \else\ifx#3\mathrel
    \Umathchardef #2 = 3 \unimathfam #1 
    \Umathcode    #1 = 3 \unimathfam #1 
  \else\ifx#3\mathopen
    \Umathcode    #1 = 4 \unimathfam #1 
    \Udelcode     #1 =   \unimathfam #1 
    \gdef#2{\Udelimiter 4 \unimathfam #1 }
    \ifx#2\sqrt
      \gdef#2{\Uradical \unimathfam #1 }
    \fi\ifx#2\cuberoot
      \gdef#2{\Uradical \unimathfam #1 }
    \fi\ifx#2\fourthroot
      \gdef#2{\Uradical \unimathfam #1 }
    \fi\ifx#2\longdivision
      \gdef#2{\Uradical \unimathfam #1 }
    \fi
  \else\ifx#3\mathclose
    \Umathcode    #1 = 5 \unimathfam #1 
    \Udelcode     #1 =   \unimathfam #1 
    \gdef#2{\Udelimiter 5 \unimathfam #1 }
  \else\ifx#3\mathpunct
    \Umathchardef #2 = 6 \unimathfam #1 
    \Umathcode    #1 = 6 \unimathfam #1 
  \else\ifx#3\mathfence % 
    \Umathchardef #2 = 0 \unimathfam #1 
    \Umathcode    #1 = 0 \unimathfam #1 
    \Udelcode     #1 =   \unimathfam #1 
    \gdef#2{\Udelimiter 0 \unimathfam #1 }
  \else\ifx#3\mathaccent
    \gdef#2{\Umathaccent fixed 0 \unimathfam #1 }
  \else\ifx#3\mathaccentwide% or overlay
    \gdef#2{\Umathaccent 0 \unimathfam #1 }
  \else\ifx#3\mathbotaccentwide
    \gdef#2{\Umathaccent bottom 0 \unimathfam #1 }
  \else\ifx#3\mathover
    \gdef#2##1{\mathop{\Umathaccent 0 \unimathfam #1 {##1}}\limits}
  \else\ifx#3\mathunder
    \gdef#2##1{\mathop{\Umathaccent bottom 0 \unimathfam #1 {##1}}\limits}
  \else\ifx#3\mathbotaccent% This type's frequency is the lowest.
    \gdef#2{\Umathaccent bottom fixed 0 \unimathfam #1 }
  \else% undefined math type
    \message{There's an undefined math type. Math character command ignored.}%
  \fi\fi\fi\fi\fi\fi\fi\fi\fi\fi\fi\fi\fi\fi
}
\input unicode-math-table

%\Umathcode `\! = 5 \unimathfam `\!
\Umathcode `\* = 2 \unimathfam `\*
%\Umathcode `\+ = 2 \unimathfam `\+
%\Umathcode `\, = 6 \unimathfam `\,
\Umathcode `\- = 2 \unimathfam "2212
%\Umathcode `\. = 0 \unimathfam `\.
\Umathcode `\: = 3 \unimathfam `\:
\Umathcode `\; = 6 \unimathfam `\;
\Umathcode `\< = 3 \unimathfam `\<
\Umathcode `\= = 3 \unimathfam `\=
\Umathcode `\> = 3 \unimathfam `\>
%\Umathcode `\? = 5 \unimathfam `\?
\Umathcode `\_ = 0 \unimathfam `\_
%\Umathcode `\| = 0 \unimathfam `\|
%\Umathcode `\/ = 0 \unimathfam `\/
%\Umathcode `\\ = 0 \unimathfam `\\
%\Umathcode `\( = 4 \unimathfam `\(
%\Umathcode `\) = 5 \unimathfam `\)
%\Umathcode `\[ = 4 \unimathfam `\[
%\Umathcode `\] = 5 \unimathfam `\]
%\Umathcode `\{ = 4 \unimathfam `\{
%\Umathcode `\} = 5 \unimathfam `\}
\Umathchardef \colon = 7 \unimathfam `\:

\Umathcode `\0 = 0 \unimathfam "30
\Umathcode `\1 = 0 \unimathfam "31
\Umathcode `\2 = 0 \unimathfam "32
\Umathcode `\3 = 0 \unimathfam "33
\Umathcode `\4 = 0 \unimathfam "34
\Umathcode `\5 = 0 \unimathfam "35
\Umathcode `\6 = 0 \unimathfam "36
\Umathcode `\7 = 0 \unimathfam "37
\Umathcode `\8 = 0 \unimathfam "38
\Umathcode `\9 = 0 \unimathfam "39
\Umathcode `\A = 0 \unimathfam "1D434
\Umathcode `\B = 0 \unimathfam "1D435
\Umathcode `\C = 0 \unimathfam "1D436
\Umathcode `\D = 0 \unimathfam "1D437
\Umathcode `\E = 0 \unimathfam "1D438
\Umathcode `\F = 0 \unimathfam "1D439
\Umathcode `\G = 0 \unimathfam "1D43A
\Umathcode `\H = 0 \unimathfam "1D43B
\Umathcode `\I = 0 \unimathfam "1D43C
\Umathcode `\J = 0 \unimathfam "1D43D
\Umathcode `\K = 0 \unimathfam "1D43E
\Umathcode `\L = 0 \unimathfam "1D43F
\Umathcode `\M = 0 \unimathfam "1D440
\Umathcode `\N = 0 \unimathfam "1D441
\Umathcode `\O = 0 \unimathfam "1D442
\Umathcode `\P = 0 \unimathfam "1D443
\Umathcode `\Q = 0 \unimathfam "1D444
\Umathcode `\R = 0 \unimathfam "1D445
\Umathcode `\S = 0 \unimathfam "1D446
\Umathcode `\T = 0 \unimathfam "1D447
\Umathcode `\U = 0 \unimathfam "1D448
\Umathcode `\V = 0 \unimathfam "1D449
\Umathcode `\W = 0 \unimathfam "1D44A
\Umathcode `\X = 0 \unimathfam "1D44B
\Umathcode `\Y = 0 \unimathfam "1D44C
\Umathcode `\Z = 0 \unimathfam "1D44D
\Umathcode `\a = 0 \unimathfam "1D44E
\Umathcode `\b = 0 \unimathfam "1D44F
\Umathcode `\c = 0 \unimathfam "1D450
\Umathcode `\d = 0 \unimathfam "1D451
\Umathcode `\e = 0 \unimathfam "1D452
\Umathcode `\f = 0 \unimathfam "1D453
\Umathcode `\g = 0 \unimathfam "1D454
\Umathcode `\h = 0 \unimathfam "0210E % Planck constant
\Umathcode `\i = 0 \unimathfam "1D456
\Umathcode `\j = 0 \unimathfam "1D457
\Umathcode `\k = 0 \unimathfam "1D458
\Umathcode `\l = 0 \unimathfam "1D459
\Umathcode `\m = 0 \unimathfam "1D45A
\Umathcode `\n = 0 \unimathfam "1D45B
\Umathcode `\o = 0 \unimathfam "1D45C
\Umathcode `\p = 0 \unimathfam "1D45D
\Umathcode `\q = 0 \unimathfam "1D45E
\Umathcode `\r = 0 \unimathfam "1D45F
\Umathcode `\s = 0 \unimathfam "1D460
\Umathcode `\t = 0 \unimathfam "1D461
\Umathcode `\u = 0 \unimathfam "1D462
\Umathcode `\v = 0 \unimathfam "1D463
\Umathcode `\w = 0 \unimathfam "1D464
\Umathcode `\x = 0 \unimathfam "1D465
\Umathcode `\y = 0 \unimathfam "1D466
\Umathcode `\z = 0 \unimathfam "1D467
\Umathcode `\Α = 0 \unimathfam "1D6E2
\Umathcode `\Β = 0 \unimathfam "1D6E3
\Umathcode `\Γ = 0 \unimathfam "1D6E4
\Umathcode `\Δ = 0 \unimathfam "1D6E5
\Umathcode `\Ε = 0 \unimathfam "1D6E6
\Umathcode `\Ζ = 0 \unimathfam "1D6E7
\Umathcode `\Η = 0 \unimathfam "1D6E8
\Umathcode `\Θ = 0 \unimathfam "1D6E9
\Umathcode `\Ι = 0 \unimathfam "1D6EA
\Umathcode `\Κ = 0 \unimathfam "1D6EB
\Umathcode `\Λ = 0 \unimathfam "1D6EC
\Umathcode `\Μ = 0 \unimathfam "1D6ED
\Umathcode `\Ν = 0 \unimathfam "1D6EE
\Umathcode `\Ξ = 0 \unimathfam "1D6EF
\Umathcode `\Ο = 0 \unimathfam "1D6F0
\Umathcode `\Π = 0 \unimathfam "1D6F1
\Umathcode `\Ρ = 0 \unimathfam "1D6F2
\Umathcode `\Σ = 0 \unimathfam "1D6F4
\Umathcode `\Τ = 0 \unimathfam "1D6F5
\Umathcode `\Υ = 0 \unimathfam "1D6F6
\Umathcode `\Φ = 0 \unimathfam "1D6F7
\Umathcode `\Χ = 0 \unimathfam "1D6F8
\Umathcode `\Ψ = 0 \unimathfam "1D6F9
\Umathcode `\Ω = 0 \unimathfam "1D6FA
\Umathcode `\α = 0 \unimathfam "1D6FC
\Umathcode `\β = 0 \unimathfam "1D6FD
\Umathcode `\γ = 0 \unimathfam "1D6FE
\Umathcode `\δ = 0 \unimathfam "1D6FF
\Umathcode `\ε = 0 \unimathfam "1D700
\Umathcode `\ζ = 0 \unimathfam "1D701
\Umathcode `\η = 0 \unimathfam "1D702
\Umathcode `\θ = 0 \unimathfam "1D703
\Umathcode `\ι = 0 \unimathfam "1D704
\Umathcode `\κ = 0 \unimathfam "1D705
\Umathcode `\λ = 0 \unimathfam "1D706
\Umathcode `\μ = 0 \unimathfam "1D707
\Umathcode `\ν = 0 \unimathfam "1D708
\Umathcode `\ξ = 0 \unimathfam "1D709
\Umathcode `\ο = 0 \unimathfam "1D70A
\Umathcode `\π = 0 \unimathfam "1D70B
\Umathcode `\ρ = 0 \unimathfam "1D70C
\Umathcode `\ς = 0 \unimathfam "1D70D
\Umathcode `\σ = 0 \unimathfam "1D70E
\Umathcode `\τ = 0 \unimathfam "1D70F
\Umathcode `\υ = 0 \unimathfam "1D710
\Umathcode `\φ = 0 \unimathfam "1D719
\Umathcode `\χ = 0 \unimathfam "1D712
\Umathcode `\ψ = 0 \unimathfam "1D713
\Umathcode `\ω = 0 \unimathfam "1D714
\Umathcode `\ϑ = 0 \unimathfam "1D717
\Umathcode `\ϕ = 0 \unimathfam "1D711
\Umathcode `\ϖ = 0 \unimathfam "1D71B
\Umathcode `\ϰ = 0 \unimathfam "1D718
\Umathcode `\ϱ = 0 \unimathfam "1D71A
\Umathcode `\ϴ = 0 \unimathfam "1D6F3
\Umathcode `\ϵ = 0 \unimathfam "1D716

\Umathcode `\Ϝ = 0 \unimathfam "003DC
\Umathcode `\ϝ = 0 \unimathfam "003DD
\Umathcode `\϶ = 0 \unimathfam "003F6

\input unimath-plain-alphafams

% Greek letter command
\let\Alpha=Α      \let\Beta=Β     \let\Gamma=Γ    \let\Delta=Δ
\let\Epsilon=Ε    \let\Zeta=Ζ     \let\Eta=Η      \let\Theta=Θ
\let\Iota=Ι       \let\Kappa=Κ    \let\Lambda=Λ   \let\Mu=Μ
\let\Nu=Ν         \let\Xi=Ξ       \let\Omicron=Ο  \let\Pi=Π
\let\Rho=Ρ        \let\Sigma=Σ    \let\Tau=Τ      \let\Upsilon=Υ
\let\Phi=Φ        \let\Chi=Χ      \let\Psi=Ψ      \let\Omega=Ω
\let\alpha=α      \let\beta=β     \let\gamma=γ    \let\delta=δ
\let\varepsilon=ε \let\zeta=ζ     \let\eta=η      \let\theta=θ
\let\iota=ι       \let\kappa=κ    \let\lambda=λ   \let\mu=μ
\let\nu=ν         \let\xi=ξ       \let\omicron=ο  \let\pi=π
\let\rho=ρ        \let\varsigma=ς \let\sigma=σ    \let\tau=τ
\let\upsilon=υ    \let\phi=φ      \let\chi=χ      \let\psi=ψ
\let\omega=ω      \let\vartheta=ϑ \let\varphi=ϕ   \let\varpi=ϖ
\let\varkappa=ϰ   \let\varrho=ϱ   \let\varTheta=ϴ \let\epsilon=ϵ

% math commands
\protected\def\{{\ifmmode\lbrace\else\char`\{\fi} % plain.tex: \let\{=\lbrace
\protected\def\}{\ifmmode\rbrace\else\char`\}\fi} % plain.tex: \let\}=\rbrace
\let\neq=\ne
\let\le=\leq
\let\ge=\geq
\let\owns=\ni
\let\gets=\leftarrow
\let\to=\rightarrow
\let\hbar=\hslash
\def\cdots{\mathinner{\cdotp\mkern-1mu\cdotp\mkern-1mu\cdotp}}
\Umathchardef \ldotp = 2 \unimathfam "2E
\def\ldots{\mathinner{\ldotp\mkern-1mu\ldotp\mkern-1mu\ldotp}}

% math and text kerns
\protected\def\leavevmode@ifvmode{\ifvmode\expandafter\indent\fi}
\protected\def\,{\ifmmode\mskip\thinmuskip\else
  \leavevmode@ifvmode\kern.16667em\fi\relax}
\protected\def\>{\ifmmode\mskip\medmuskip\else
  \leavevmode@ifvmode\kern.2222em\fi\relax}
\protected\def\;{\ifmmode\mskip\thickmuskip\else
  \leavevmode@ifvmode\kern.2777em\fi\relax}
\protected\def\!{\ifmmode\mskip-\thinmuskip\else
  \leavevmode@ifvmode\kern-.16667em\fi\relax}
\def\*{\discretionary{\thinspace\hbox{$\times$}}{}{}}

% text commands
\chardef\ss="00DF
\chardef\ae="00E6
\chardef\oe="0153
\chardef\o="00F8
\chardef\AE="00C6
\chardef\OE="0152
\chardef\O="00D8
\chardef\i="0131 \chardef\j="237 % dotless letters
\chardef\aa="00E5
\chardef\l="0142
\chardef\L="0141
\chardef\AA="00C5
\protected\def\dag{\ifmmode\mathord{\dagger}\else\char"2020\fi}
\protected\def\ddag{\ifmmode\mathord{\ddagger}\else\char"2021\fi}
\chardef\S="00A7
\chardef\P"00B6
\chardef\copyright="00A9

\catcode`\@=12
\tenrm
\endinput